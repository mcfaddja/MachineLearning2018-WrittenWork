\section{Introduction}

With the explosion in data collection/production by social media sites and their associated accounts, it has become desirable to predict attributes of many users based an analysis of a small subset of users.  The goal of this project was to accomplish this in a small way by predicting a user?s

\begin{itemize}
	\item Age Group
	\item Gender
	\item	Scores on a 5-part personality test
\end{itemize}

The ages were grouped using the following ranges

\begin{itemize}
	\item 24 \& under
	\item 25-34
	\item 35-49
	\item 50 \& over
\end{itemize}

The personality scores measured five components of a user?s personality. The users had their personality traits of

\begin{itemize}
	\item Openness (OPE)
	\item Extraversion (EXT)
	\item Conscientiousness (CON)
	\item Neuroticism (NEU)
	\item Agreeableness (AGR)
\end{itemize}

scored on a scale from 1 to 5, with a resolution to the hundredths place.

The goals were to predict the described user attributes using models which performed better than a baseline model.  These models were compared using accuracy for the age and gender attributes and root mean squared error for the personality attributes.  Baseline scores to beat were 0.59 for both age \& gender, 0.65 for openness, 0.79 for extroversion,  0.80 for neuroticism, 0.73 for conscientiousness, and 0.66 for agreeableness.  Compared to these, we obtained the following results against our own test data are given in \textbf{Table \ref{intro:ownDATA}}.

\begin{table}
\caption{RESULTS Against Seen Data}
\label{intro:ownDATA}
\begin{minipage}{\columnwidth}
	\begin{center}
	\begin{tabular}{l | c | c | c }
  	\toprule
				& \textbf{Likes}		& \textbf{Images}	& \textbf{Text}		\\
		\hline
		Age		& 0.65			& 				& 				\\
		Gender	& 0.81			& 				& 				\\
		OPE		& 0.60			&  				& 				\\
		EXT		& 0.78			&  				& 				\\
		CON		& 0.65			&  				& 				\\
		NEU		& 0.75			&  				& 				\\
		AGR		& 0.64			&  				& 				\\
	\bottomrule
	\end{tabular}
	\end{center}
\bigskip
\end{minipage}
\end{table}

Against unseen test data, the results we obtained are given in \textbf{Table \ref{intro:hideDATA}}.

\begin{table}
\caption{RESULTS Against Hidden Data}
\label{intro:hideDATA}
\begin{minipage}{\columnwidth}
	\begin{center}
	\begin{tabular}{l | c | c | c }
  	\toprule
				& \textbf{Likes}		& \textbf{Images}	& \textbf{Text}		\\
		\hline
		Age		& 0.67			& 				& 				\\
		Gender	& 0.83			& 				& 				\\
		OPE		& 0.65			&  				& 				\\
		EXT		& 0.77			&  				& 				\\
		CON		& 0.32			&  				& 				\\
		NEU		& 0.80			&  				& 				\\
		AGR		& 0.67			&  				& 				\\
	\bottomrule
	\end{tabular}
	\end{center}
\bigskip
\end{minipage}
\end{table}


%---------------------------------------------------
%---------------------------------------------------



\section{Methodology}

Since there were three different types of data, we have divided describing our methodology into three parts.  Each data type/source (\emph{text, images, \& likes}) will have its own part (\emph{section}).

\subsection{Text}

For the Text part, we separate it into three parts. The first one is using the text to predict the accuracy of age and gender, then we use the 82- LIWC features to predict the big five personalities. For each part, we also use different methods to build the basic model to test accuracy.

We tried four different methods in text part, Linear regression, Random forest, Na�ve Bayes and Logistical regression. Finally, we find the best result from the method which uses logistical regression and Na�ve regression.

The first step of the text part is data preprocessing. In the files, we find that they are separated, thus we merge the profile file and text file according to the common primary key in each take. We add each text part into profile.csv. At the meanwhile, we also transfer the age to age groups, and we use 1 and 0 to replace the female and male in order to more effective.

After data preprocessing, we use the Na�ve Bayes to predict the age and gender. As is known to us who focus on machine learning. Na�ve Bayes classifiers are a family of simple "probabilistic classifiers" based on applying Bayes theorem with strong independence assumptions between the features. However, we find the model only working well on gender part. The accuracy of age in Na�ve Bayes model is 0.49 which is lower than baseline (0.59). Therefore, we have to change our method. After researching, we guess the Logistical Regression will work well on age part. At the beginning, we split the training data into two part, the one is training data which are 8000 rows, the other is test data which is 1500 rows. To our surprise, it does not work well. After trying different methods, like linear regression, random forest. We find the model cannot work well when we use text to predict the age, thus we try to use 82-LIWC features to build model.

The LIWC, The linguistic Inquiry and Word Count tool, is known text analysis software which is widely used in psychology studies, In the file, for each user, it has 82 features. We have to merge the LIWC and profile file. When we create the model, we delete the big five personalities which need to be predicted in the table. After testing in local machine, we find it work well in age predict. The accuracy of age group reach to 0.62.

For Five Big Personality, we have to used 82-LIWC features. Because of big data, we decide to use Linear regression to set the model. In statistics, Linear regression is a linear approach to modeling the relationship between a scalar response and one or more explanatory variables. Fortunately, Linear regression work well in the model, the results almost similar with baseline. Although we try to decrease the RMSE of five big personality, the methods we tried are almost same. 


%---------------------------------------------------

\subsection{Images}

Convolutional Neural Networks was used to handle the image source of this experiment. CNN is a class of deep, feed forward ANN having applications in image/video processing, NLP etc. CNNs draw inspiration the animal vision cortex system. CNNs requires minimal preprocessing, i.e. the network learns the features as opposed to them being engineered. CNNs work by extracting features and recognizing patterns from the images, by convoluting the image with filters. 
CNNs have proved to be a robust and low error approach for the image classification tasks. The overlap between Machine Learning and Computer Vision has seen many recent technological developments, particularly Deep Learning. CNNs with deeper/wider layers and larger training datasets tend to perform better. 

In order to avoid over-fitting, data augmentation was performed on the image database. The augmentation operations included rescaling each pixel, flipping horizontally, rotating the image by variable degrees, randomly zooming into the image etc. 

A series of experiments was conducted, with different CNN models, before and after performing any filtering operations through Haar cascades. Only gender prediction was implemented with images, since experiments with age prediction did not give good results. 

%
\subsubsection{Image Model-1: Basic CNN Model}

Reference for this model was taken from the Keras blog\footnote{\textbf{Reference:} https://blog.keras.io/building-powerful-image-classification-models-using-very-little-data.html}.  The model had an input size of 224x224 pixels, with 3 convolutional layers and and one fully connected layer. The optimizer used was 'adam' and the the evaluation metric was 'accuracy'.  This model is described in \textbf{Table \ref{tab:img-model1}}.

\begin{table}
\caption{IMAGE MODEL-1: Basic CNN Model}
\label{tab:img-model1}
\begin{minipage}{\columnwidth}
	\begin{center}
	\begin{tabular}{l | c | c | c | c | c}
  	\toprule
		\textbf{Layer}		& \textbf{Filters}		& \textbf{Neurons}	& \textbf{Fltr. Sz}	& \textbf{Act.}	& \textbf{Dropout}	\\
		\hline
		Conv	\footnote{Convolutional}	& 32					& -					& $3 \times 3$		& relu			& -				\\
		Conv	\footnote{Convolutional}	& 32					& -					& $3 \times 3$		& relu			& -				\\
		Conv	\footnote{Convolutional}	& 64					& -					& $3 \times 3$		& relu			& -				\\
		Full-Con\footnote{Fully Connected}	& -					& 64					& -				& relu			& 0.5				\\
		Output			& -					& 1					& -				& sigmoid			& -				\\
	\bottomrule
	\end{tabular}
	\end{center}
\bigskip
\footnotesize
  \emph{Note:} Fltr. Sz = Filter Size \\
  \emph{Note:} Act. = Activation Function
\end{minipage}
\end{table}

%
\subsubsection{Image Model-2: RESNET50}

Bottleneck features were extracted from the RESNET50 model. Final layer of the model was not included. The pre-trained weights were used to find the bottleneck values, i.e. features were extracted. Fully connected layers were added to run with the training data. This model is described in \textbf{Table \ref{tab:img-model2}}.

\begin{table}
\caption{IMAGE MODEL-1: Basic CNN Model}
\label{tab:img-model2}
\begin{minipage}{\columnwidth}
	\begin{center}
	\begin{tabular}{l | c | c | c | c | c}
  	\toprule
		\textbf{Layer}					& \textbf{Filters}		& \textbf{Neurons}	& \textbf{Fltr. Sz}	& \textbf{Act.}	& \textbf{Dropout}	\\
		\hline
		Full-Con\footnote{Fully Connected}	& -					& 256			& -				& relu			& 0.5		\\
		Output						& -					& 1				& -				& sigmoid			& -		\\
	\bottomrule
	\end{tabular}
	\end{center}
\bigskip
\footnotesize
  \emph{Note:} Fltr. Sz = Filter Size \\
  \emph{Note:} Act. = Activation Function
\end{minipage}
\end{table}



%---------------------------------------------------

\subsection{Likes}

When considering the Likes data, we first create a \textbf{User/LikeID Matrix} from the training data we are given.  This process is described below in the Dataset and Metrics section. Based on the training data we were given, this resulted in a matrix with 536204 columns.  Due to the large number of columns in this matrix, dimensionality reduction using Singular Value Decomposition or Principal Components Analysis was considered.  Unfortunately, when the singular values of this matrix were computed (\emph{up to the 9499th singular value}), there was no characteristic "cliff" drop-off in their magnitudes which would indicate a good point for truncation.  

To illustrate this, the type of "cliff" in the singular value magnitudes we were seeking can be seen in the plot of singular value magnitudes, from a different dataset given in \textbf{Figure 1}.

\begin{figure}
\includegraphics[width=3.5in]{cliffWANTEDa}
\caption{The type of 'cliff' drop-off in singular value magnitudes we wanted to see.}
\end{figure}

Unfortunately, the result was got did not have the sudden drop over just a few singular values.  What we got instead can be seen in \textbf{Figure 2}.

\begin{figure}
\includegraphics[width=3.5in]{cliffGOT}
\caption{The singular value magnitudes we got instead of what we wanted to see.}
\end{figure}

Thus, we proceeded to use the entirety of the column space of the \textbf{User/LikeID Matrix}.  We constructed models using the machine learning algorithms

\begin{itemize}
	\item	Random Forest
	\item AdaBoost
	\item Bagging (\emph{with in-bag scoring})
	\item Bagging (\emph{with out-of-bag scoring})
	\item Naive Bayes
	\item	K-Nearest Neighbor
	\item	Support Vector Machine
	\item Linear Support Vector Machine
	\item Gradient Boosting
	\item Neural Networks (\emph{implemented using Keras})
\end{itemize}

Each of these algorithms was run using a variety of performance parameters, including the number of estimators, the learning rate, the tolerance, the number of neighbors, and more.  Ultimately, for each of the user attributes, we chose the algorithms and settings

\begin{itemize}
	\item \textbf{\underline{Age Group}: Neural Network} with \textbf{750 base nodes} (\emph{see below for full network design} \\
	\item \textbf{\underline{Gender}: Neural Network} with \textbf{650 base nodes} (\emph{see below for full network design} \\
	\item \textbf{\underline{OPE}: SVM} with \textbf{tolerance of $1 \times 10^{-6}$} \\
	\item	 \textbf{\underline{NEU}: AdaBoost} with \textbf{100 estimators and a learning rate of 1.0} \\
	\item	 \textbf{\underline{EXT}: Random Forest} with \textbf{1500 estimators} \\
	\item	 \textbf{\underline{AGR}: Bagging \emph{using in-bag scoring}} with \textbf{250 estimators} \\
	\item	 \textbf{\underline{CON}: Bagging \emph{using in-bag scoring}} with \textbf{100 estimators}
\end{itemize}

We also strongly considered the alternative algorithms and settings

\begin{itemize}
	\item	 \textbf{\underline{NEU}: AdaBoost} with \textbf{250 estimators and a learning rate of 0.01} \\
	\item	 \textbf{\underline{EXT}: K-Nearest Neighbors} with \textbf{500 neighbors} \\
	\item	 \textbf{\underline{AGR}: K-Nearest Neighbors} with \textbf{800 neighbors} \\
	\item	\textbf{\underline{CON}: AdaBoost} with \textbf{1000 estimators and a learning rate of 1.0}
\end{itemize}

%
\subsubsection{Neural Net Design}

We designed three types of neural networks for this part of the project.  The first type of neural net was a single-class binary classifier for predicting gender; while second type as a multi-class binary classifier for predicting age group.  Finally, the last type of neural net we designed was a multilayer linear regressor.  All three networks had the same number of hidden and dropout layers, however they used different activation functions and kernel initializers.  Additionally, all of these neural networks were implemented in \textbf{Keras} using its \textbf{Sequential} model.  We summarize the three neural nets in \textbf{Table \ref{tab:like-nnets}}.

\begin{table}
\caption{Likes SUMMARY of Neural Nets}
\label{tab:like-nnets}
\begin{minipage}{\columnwidth}
	\begin{center}
	\begin{tabular}{l | c | c | c }
  	\toprule
						& \textbf{Gender}	& \textbf{Age Group}		& \textbf{Regressor}	\\
		\hline
		\hline
		Input Layer		&				&					&				\\
		$\qquad$ Neurons	& 536204			& 536204				& 536204			\\
		\hline
		1st Hidden Layer	&				&					&				\\
		$\qquad$ Dense	& 975			& 1125				& 750			\\
		$\qquad$ Kernel	& \emph{uniform}	& \emph{default}		& \emph{default}	\\
		$\qquad$ Activation	& \emph{relu}		& \emph{relu}			& \emph{relu}		\\
		\hline
		1st Dropout Layer	& 0.25			& 0.25				& 0.25			\\
		\hline
		2nd Hidden Layer	&				&					&				\\
		$\qquad$ Dense	& 1300			& 1500				& 1000			\\
		$\qquad$ Kernel	& \emph{uniform}	& \emph{default}		& \emph{default}	\\
		$\qquad$ Activation	& \emph{relu}		& \emph{softmax}		& \emph{relu}		\\
		\hline
		2nd Dropout Layer	& 0.375			& 0.375				& 0.375			\\
		\hline
		3rd Hidden Layer	&				&					&				\\
		$\qquad$ Dense	& 975			& 1125				& 750			\\
		$\qquad$ Kernel	& \emph{uniform}	& \emph{default}		& \emph{default}	\\
		$\qquad$ Activation	& \emph{sigmoid}	& \emph{sigmoid}		& \emph{sigmoid}	\\
		\hline
		3rd Dropout Layer	& 0.25			& 0.25				& 0.25			\\
		\hline
		4th Hidden Layer	&				&					&				\\
		$\qquad$ Dense	& 650			& 750				& 500			\\
		$\qquad$ Kernel	& \emph{uniform}	& \emph{default}		& \emph{default}	\\
		$\qquad$ Activation	& \emph{relu}		& \emph{relu}			& \emph{sigmoid}	\\
		\hline
		Output Layer		&				&					&				\\
		$\qquad$ Neurons	& 1				& 4					& 1				\\
		$\qquad$ Kernel	& \emph{uniform}	& \emph{default}		& \emph{default}	\\
		$\qquad$ Activation	& \emph{sigmoid}	& \emph{softmax}		& \emph{sigmoid}	\\
	\bottomrule
	\end{tabular}
	\end{center}
\bigskip
\end{minipage}	
\end{table}



%---------------------------------------------------
%---------------------------------------------------



\section{Dataset and Metrics}

Since the dataset was divided into three classes of data, we describe the datasets provided for each data class in its own, separate section below.  However, the metrics were similar for all classes of data.  The age and gender attributes were categorical, so we used the accuracy of their prediction as our metric for models predicting those attributes.  For the personality attributes, we used root-mean-squared-error as the metric for any model predicting those attributes. \\

Additionally, the one dataset common to all approaches was the profile data.  That data had 9500 users and gave their

\begin{itemize}
	\item Age (\emph{as a double all ending in $.0$})
	\item Gender (\emph{with 0 for male and 1 for female, as a double all ending in $.0$})
	\item	Openness (\emph{as a double on a scale from 1 to 5})
	\item Extroversion (\emph{as a double on a scale from 1 to 5})
	\item Conscientiousness (\emph{as a double on a scale from 1 to 5})
	\item Neuroticism (\emph{as a double on a scale from 1 to 5})
	\item Agreeableness (\emph{as a double on a scale from 1 to 5})
\end{itemize}


\subsection{Text}

There are five filed in the training dataset. LIWC, profile, relation and test. For text and LIWC part, we use the text folder which contain 9500 text files, and we merge text files and a csv file named "profile.csv" because of common primary key "userid". Meanwhile, we merge the profile and the LIWC file as well. In the profile, we use 0 denote male, and 1 is female. In the age, we have classified the age likes "xx-24", "25-34", "35-69", and "50-xx". In order to more effective, the big five personality, Openness, Conscientiousness, Extroversion, Agreeableness, Emotional Stability, being referred to as "ope", "con", "ext", "agr" and "neu".


%---------------------------------------------------

\subsection{Likes}

The initial likes dataset contained one column with User IDs and one column with Like IDs, where the User IDs values were repeated several times to indicate all of the posts that a given user had liked.  Overall, the initial input file contained over 1.5 million different 'likes'.  To make the data easier for machine learning algorithms to process this initial data was converted into a sparse matrix with 9500 rows and 536204 columns, with each row representing a user and each column representing a liked item.  If an arbitrary user had liked any single item, then the element in the row for that user and the column for that liked item would have a 1 for its value.  Any items not liked by a user would have 0 as the value for the element in the row for that user and the column for that liked item.  This is why the sparse matrix was so useful.  There were only ~1.67 million likes in a matrix with 5,098,538,000 elements, so most ($0.03$ \%) of the matrix was empty (\emph{i.e. zero valued}).  Performing a dimensional reduction on this matrix using SVD or PCA would have been useful for reducing computation load and time, as well as opening up more complex machine learning algorithms; however, as discussed in the Likes subsection of the Methodology section (2.3), no good truncation point was found.

Aside from the overall metrics discussed in the opening of this section we also employed entropy loss when examining the age and gender attributes.  Additionally, categorizing the personality scores by 0.1 (\emph{yielding 40 categories}) was considered but ultimately rejected as too inaccurate.  Finally, when considering the personality scores as continuous variables between 1 and 5 (\emph{inclusive}), we used \textbf{Mean-Sqared-Error} and the \textbf{$R^2$ coefficient} as additional metrics to examine model performance.


%---------------------------------------------------

\subsection{Images}

The image database contained 9500 images, which was split into 8000 for training and 1500 for validation. For the purposes of gender prediction, the 8000 images were regrouped according to the labels, i.e. 3386 males and 4614 females. The regrouping was done using the profile.csv file. One critical issue to note is that it is an unbalanced distribution between the two labels. 
Some of the issues noticed in the database were as follows:

\begin{itemize}
	\item	Images did not contain faces
	\item	Images contained multiple faces
	\item Images did not contain face belonging to the label holder
\end{itemize}

The third issue mentioned above cannot be tackled, as the system lacks that intelligent ability. Operations can be implemented to filter the images with the first two issues mentioned above. OpenCV Haar Cascades were used as the filters. Haar cascades is a method used for object detection in images. There are specialized .xml files present, for detection of various objects. These files are created by a machine learning approach where a cascade function is trained from a lot of positive and negative images. 
The cascade functions customized for face detection were used in this project, to filter images which had multiple faces or images which did not contain faces. The cascade functions used were for different views of a face, i.e. frontal and profile views. The images which passed through at least one cascade were retained. Manual selection was performed on the few residual images, to perform a final filtering operation. 

%
\subsubsection{Image Metrics}

This project contains usage of various machine learning algorithms, which are either classification or regression. The evaluation metric of 'accuracy' was used for the classification models, and 'root mean squared error' for the regression models. 




%---------------------------------------------------
%---------------------------------------------------


\section{Results}

Our overall results can be summarized by the information in \textbf{Table \ref{intro:ownDATA}} and \textbf{Table \ref{intro:hideDATA}}, given above.  Detailed results for the three different classes of data are given in their own sections below.


%---------------------------------------------------

\subsection{Text}

For age prediction model, using Logistical regression based on 82-LIWC features is best choice in the method we tried. We also use other methods, line Linear regression, random forest, Na�ve Bayes. We can see the results in the chart given in \textbf{Table \ref{tab:txt-ageSEX}}.

\begin{table}
\caption{Text RESULTS for Age \& Gender Accuracy}
\label{tab:txt-ageSEX}
\begin{minipage}{\columnwidth}
	\begin{center}
	\begin{tabular}{l | c | c}
  	\toprule
							& \textbf{Age Group}				& \textbf{Gender}	\\
		\hline
		\hline
   		Baseline				& 0.59						& 0.59			\\
		\hline
		Linear Regression		& 0.52						& 0.60			\\
		\hline
		Random Forest			& 0.51						& 0.60			\\
		\hline
		Naive Bayes			& 0.49						& 0.71			\\
		\hline
		Logistical Regression	& 0.61\footnote{Using LIWC data}	& 0.59			\\
							& 0.59\footnote{Using text data}	& 				\\
	\bottomrule
	\end{tabular}
	\end{center}
\bigskip
\end{minipage}	
\end{table}

From the chart in \textbf{Table \ref{tab:txt-ageSEX}}, we can see that the best result in age prediction is the model when it uses logistical regression based on LIWC, and the best accuracy in gender predict model is Na�ve Bayes. Therefore, our choice are Na�ve Bayes based on text in gender and Logistical Regression based on LIWC in age prediction.

For Big five personality, we only use linear regression on the text and LIWC part, because we tried other methods, and the results are same. The results are in the chart given in \textbf{Table \ref{tab:txt-person}}.

\begin{table}
\caption{Text RESULTS for Personality Score RMSE}
\label{tab:txt-person}
\begin{minipage}{\columnwidth}
	\begin{center}
	\begin{tabular}{l | c | c | c | c | c}
  	\toprule
						& \textbf{OPE}	& \textbf{NEU}	& \textbf{EXT}	& \textbf{AGR}	& \textbf{CON}	\\
		\hline
		Baseline			& 0.65		& 0.80		& 0.79		& 0.66		& 0.73		\\
		Linear Regression	& 0.65		& 0.79		& 0.79		& 0.65		& 0.72		\\
	\bottomrule
	\end{tabular}
	\end{center}
\bigskip
\end{minipage}
\end{table}

From the chart in \textbf{Table \ref{tab:txt-person}}, we can clearly see that the results under the predict model are similar with Baseline when we use Linear regression.


%---------------------------------------------------

\subsection{Images}

The basic CNN model improves the gender prediction from 0.59 to 0.66. The CNN model is trained from scratch. Further improvement was shown by cleaning the data and augmenting it, where images were filtered with Haar cascades, and augmentation was performed by internal libraries of Keras. Work with RESNET50 was attempted, however the status of the experiment is in progress.
Using Haar cascades improves the performance, as it removes a lot of bias from the dataset. The accuracy curves of a few of the experiments is pasted below:



%---------------------------------------------------

\subsection{Likes}

When considering only the data visible to us, the performance of various machine learning algorithms and models on the Likes data is summarized in \textbf{Table \ref{tab:likesRESa}} and \textbf{Table \ref{tab:likesRESb}}.

\begin{table}
\caption{Text RESULTS for Personality Score RMSE}
\label{tab:likesRESa}
\begin{minipage}{\columnwidth}
	\begin{center}
	\begin{tabular}{l | c | c | c | c | c | c | c}
  	\toprule
							& Gender		& Age	& OPE	& NEU	& EXT	& AGR	& CON	\\
		\textbf{Baseline}		& 0.59		& 0.56	& 0.65	& 0.80	& 0.79	& 0.66	& 0.73	\\
		\textbf{Naive Bayes}		& 0.66		& 0.65	& 0.78	& 0.82	& 0.81	& 0.70	& 0.75	\\
		\textbf{Random Forest}	& 0.75		& 0.61	& 0.62	& 0.81	& 0.77	& 0.65	& 0.70	\\
		\textbf{AdaBoost}		& 0.61		& 0.73	& 0.70	& 0.75	& 0.80	& 0.70	& 0.74	\\
		\textbf{Bagging IN}		& 0.73		& 0.62	& 0.65	& 0.80	& 0.77	& 0.64	& 0.65	\\
	\bottomrule
	\end{tabular}
	\end{center}
\bigskip
\end{minipage}
\end{table}

\begin{table}
\caption{Text RESULTS for Personality Score RMSE}
\label{tab:likesRESb}
\begin{minipage}{\columnwidth}
	\begin{center}
	\begin{tabular}{l | c | c | c | c | c | c | c}
  	\toprule
							& Gender		& Age	& OPE	& NEU	& EXT	& AGR	& CON	\\
		\textbf{Baseline}		& 0.59		& 0.56	& 0.65	& 0.80	& 0.79	& 0.66	& 0.73	\\
		\textbf{Bagging OUT}	& -			& -		& 0.60	& 0.79	& 0.77	& 0.67	& 0.68	\\
		\textbf{SVM}			& 0.57		& 0.67	& 0.60	& 0.81	& 0.80	& 0.66	& 0.74	\\
		\textbf{Linear SVM}		& 0.77		& 0.66	& -		& -		& -		& -		& -		\\
		\textbf{Neural Network}	& 0.80		& 0.66	& $>$ 1	& $>$ 1	& $>$ 1	& $>$ 1	& $>$ 1	\\
	\bottomrule
	\end{tabular}
	\end{center}
\bigskip
\end{minipage}
\end{table}


%---------------------------------------------------
%---------------------------------------------------


\section{Conclusion}

ljhgljhg


\subsection{Text}

For text part, we have tried many methods when we set up the model, but we do not have do more specific work on data preprocessing, such as feature selection, word choice. When we predict the age and gender, we can find more significant words, like she, he which will affect our results. Although we get a good result in the age prediction when we use Logistical regression based on 82 LIWC features, it is not perfect. In our last assignment, we find the neural network may be good at decrease RMSE in big five personality, thus we can try it in the future work. 


%---------------------------------------------------

\subsection{Images}

Further work on Images includes experimenting with a few more pre-trained models, like Inception, VGG19 etc. Experiments towards age prediction with an extensive data cleaning are in the pipeline.


%---------------------------------------------------

\subsection{Likes}

For binary, single-class categorical data (\emph{Gender}), the chosen model performed above baseline and ultimately exceeded the baseline score considerably.  Similarly, the chosen model for binary, multi-class categorical data (\emph{Age Groups}), preformed above baseline; however, it did not out perform the baseline to the same extent as the model for the binary, single-class categorical data.  

When considering the continuous data, model performance was disparate.  Some model performed above baseline (\emph{although not to the extent of either the gender or age models}), while others were at or barely above baseline.  Given this disparate performance, it is possible that the loss of accuracy from categorizing the personality attributes using a conversion to a categorical system using increments of 0.1 from 1 to 5 may not produce less accurate results than treating the variables as continuous.  

For models of all attributes, recasting the models with a dimensional reduction my produce better results.  Examining the frequency counts of liked items may yield insight into a good truncation point.  After dimensional reduction, the resulting data should be run in models both before and after it has been scaled to some standard interval.  Furthermore, for models of the personality attributes, categorizing using increments of 0.1 or 0.05 from 1 to 5 should be considered, both with and without dimensional reduction.  We believe that the key to improving the quality of results from models relying on likes data lies in some sort of dimensional reduction.







































































